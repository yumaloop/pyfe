ss[dvipdfmx,autodetect-engine]{jsarticle}
\usepackage{tikz}
\usepackage{url}
\usepackage{hyperref}
\usepackage{amsmath,amssymb}
\usepackage{comment}

% see https://doratex.hatenablog.jp/entry/20180503/1525338512

\title{フィナンシャルエンジニアリング特論第2\\中間レポート}
\author{内海佑麻\footnote{連絡先 Email: uchiumi@ailab.ics.keio.ac.jp} \\
    (学籍番号: 82018398)
        \and 澤屋敷友一\footnote{連絡先 Email: yashiki@keio.jp} \\ 
            (学籍番号: 82019220)}
            \date{November 27, 2020}

            \begin{document}

            \maketitle

            \begin{abstract}
            ポートフォリオ選択モデル(1期間モデル)を構築し,それを用いて資産運用を行う金融商品を組成し,そのパフォーマンスを検証した.
            \end{abstract}

            \tableofcontents

            \newpage

            \section{モデルの構築方法}

            \subsection{ポートフォリオ選択モデル}

            \subsubsection{Markowitzの平均分散モデル}

            Markowitzの平均分散モデルでは,「ポートフォリオの期待収益率(Expected return)が一定値以上となる」という制約条件の下で,「ポートフォリオの分散を最小化する」最適化問題を考える.一般に,$n$コの資産で構成されるポートフォリオの場合,ポートフォリオの分散は$n$コの資産間の共分散行列の二次形式となるので,この最適化問題は二次計画問題(Quadratic Programming, QP)のクラスとなり,次のように定式化される.
            \begin{align}
                \underset{\bf x \in \mathcal{X}}{\rm minimize} ~~~ 
                    &\sigma_p \left( = {\bf x}^{\mathrm{T}} \Sigma {\bf x} \right) \\
                        {\rm subject~to} ~~~ &\bar{r}_p = \bar{\bf r}^{\mathrm{T}} {\bf x} = \sum_{i=1}^{n} \bar{r}_i x_i \geq r_e \\
                            &{\| {\bf x} \|}_{1} = \sum_{i=1}^{n} x_i = 1 \\
                                &x_i \geq 0 ~~ (i = 1, \cdots, n)
                                \end{align}

                                \begin{itemize}
                                    \item $\Sigma \in \mathbb{R}^{n \times n}$ - $n$コの資産の共分散行列
                                        \item ${\bf x} \in \mathbb{R}^{n}$ - $n$コの資産の投資比率ベクトル
                                            \item $\bar{\bf r} \in \mathbb{R}^{n}$ - $n$コの資産の期待収益率ベクトル
                                                \item $x_i \in \mathbb{R}$ - 資産$i$の投資比率
                                                    \item $\bar{r}_i \in \mathbb{R}$ - 資産$i$の期待収益率
                                                        \item $r_e \in \mathbb{R}$ - 投資家の要求期待収益率
                                                            \item $\bar{r}_p \in \mathbb{R}$ - ポートフォリオの期待収益率
                                                                \item $\sigma_p \in \mathbb{R}$ - ポートフォリオの標準偏差
                                                                \end{itemize}
                                                                1つ目の制約式は,ポートフォリオの期待収益率が一定値($=r_e$)以上となることを要請している.2つ目,3つ目の制約式はポートフォリオの定義からくる自明なものである.資産の空売りを許す場合,3つ目の制約式を除くこともある.

                                                                \subsubsection{Sharpe-ratio最大化モデル}

                                                                シャープレシオ(Sharpe ratio, SR)は,最もよく使われるポートフォリオに対するリスク調整済みパフォーマンス尺度である.あるポートフォリオのシャープレシオ${\theta}_p$は,無リスク資産の収益率$r_f$とポートフォリオの期待収益率$\bar{r}_p$,ポートフォリオの標準偏差$\sigma_p$を用いて
                                                                \begin{align}
                                                                    {\theta}_p := \frac{\bar{r}_p - r_f}{\sigma_p}
                                                                    \end{align}
                                                                    と定義される.シャープレシオ最大化問題は,
                                                                    目的関数に$n$コの資産間の共分散行列が含まれるため,二次計画問題(Quadratic Programming, QP)のクラスとなり,次のように定式化される.

                                                                    \begin{align}
                                                                        \underset{\bf x \in \mathcal{X}}{\rm maximize} ~~~ 
                                                                            &\frac{\bar{r}_p - r_f}{\sigma_p}
                                                                                \left( 
                                                                                    = \frac{\bar{\bf r}^{\mathrm{T}} {\bf x} - r_f}{\sqrt{{\bf x}^{\mathrm{T}} \Sigma {\bf x}}}
                                                                                        \right) \\
                                                                                            {\rm subject~to} ~~~ 
                                                                                                &{\| {\bf x} \|}_{1} = \sum_{i=1}^{n} x_i = 1 \\
                                                                                                    &x_i \geq 0 ~~ (i = 1, \cdots, n)
                                                                                                    \end{align}

                                                                                                    \begin{itemize}
                                                                                                        \item $r_f \in \mathbb{R}$ - 無リスク資産の収益率
                                                                                                        \end{itemize}

                                                                                                        この最適化問題の目的関数を二次形式で表すため,リスクプレミアム$\lambda$と各資産の比重ベクトル${\bf w} = {\bf x} / \lambda$を導入して変形すると,次のようにSharpe-ratio最大化問題のコンパクト分解表現を得る.

                                                                                                        \begin{align}
                                                                                                            \underset{\bf w \in \mathcal{W}}{\rm minimize} ~~~ 
                                                                                                                & {\bf w}^{\mathrm{T}} \Sigma {\bf w} \\
                                                                                                                    {\rm subject~to} ~~~ 
                                                                                                                        &{\| \hat{\bf r}^{\mathrm{T}} {\bf w} \|}_{1} = \sum_{i=1}^{n} (\bar{r}_i - r_f) w_i = 1 \\
                                                                                                                            &w_i \geq 0 ~~ (i = 1, \cdots, n)
                                                                                                                            \end{align}

                                                                                                                            \begin{itemize}
                                                                                                                                \item $x_i \in \mathbb{R}$ - 資産$i$の投資比率
                                                                                                                                    \item $w_i \in \mathbb{R}$ - 資産$i$の投資比率と期待リスクプレミアムの比率 ($w_i = x_i / \lambda $)
                                                                                                                                        \item $\lambda \in \mathbb{R}$ - ポートフォリオの期待リスクプレミアム ($\lambda = \sum_{i=1}^{n} \hat{r}_i x_i = \sum_{i=1}^{n} (\bar{r}_i - r_f) x_i$)
                                                                                                                                            \item $\hat{r}_i \in \mathbb{R}$ - 資産$i$の期待リスクプレミアム ($\hat{r}_i = \bar{r}_i - r_f$)
                                                                                                                                            \end{itemize}

                                                                                                                                            ここで,定義式$x_i = \lambda w_i$と予算制約式$\sum_{i=1}^{n} x_i = 1$に注意すると,$\lambda = 1/(\sum_{i=1}^{n} w_i)$となるから,元問題の実行可能解$\bf x$とコンパクト分解表現の実行可能解$\bf w$の間に以下の関係が成り立つ.

                                                                                                                                            \begin{align}
                                                                                                                                                {\bf x} = \lambda {\bf w} =  \frac{\bf w}{\sum_{i=1}^{n} w_i} = \frac{\bf w}{{\|{\bf w}\|}_1}
                                                                                                                                                \end{align}


                                                                                                                                                \subsection{ソフトウェアとデータ}

                                                                                                                                                CVXOPTの使い方

                                                                                                                                                Pythonの凸最適化向けパッケージCVXOPT (\url{https://cvxopt.org})を使って,この二次計画問題(QP)を解く.
                                                                                                                                                CVXOPTで二次計画問題を扱う場合は,解きたい最適化問題を以下の一般化されたフォーマットに整理して,

                                                                                                                                                \begin{align}
                                                                                                                                                \underset{\bf x}{\rm minimize} ~~~ 
                                                                                                                                                &\frac{1}{2} {\bf x}^{T} P {\bf x} + {\bf q}^{T} {\bf x} \\
                                                                                                                                                {\rm subject~to} ~~~ & G {\bf x} \leq {\bf h} \\
                                                                                                                                                &A {\bf x} = {\bf b}
                                                                                                                                                \end{align}

                                                                                                                                                パラメータ$P,q,G,h,A$を計算し,$cvxopt.solvers.qp()$関数を実行することで最適解と最適値を求める.Markowitzの平均・分散モデルの場合は,

                                                                                                                                                \begin{align}
                                                                                                                                                    P = 2 \cdot \Sigma, \hspace{1em}
                                                                                                                                                        q = {\bf 0}, \hspace{1em}
                                                                                                                                                            G = - {\bf I}\_n, \hspace{1em}
                                                                                                                                                                h = - r_e, \hspace{1em}
                                                                                                                                                                    A = {\bf 1}\_n^T, \hspace{1em}
                                                                                                                                                                        b = 1
                                                                                                                                                                        \end{align}

                                                                                                                                                                        となる.\footnote{参考URL: \url{https://cvxopt.org/userguide/coneprog.html#quadratic-programming}}

                                                                                                                                                                        \subsection{パラメータの設定}

                                                                                                                                                                        リバランス,パラメータの推定期間

                                                                                                                                                                        \section{バックテストとパフォーマンス評価}

                                                                                                                                                                        \section{結論と考察}

                                                                                                                                                                        \end{document}
